%%%%%%%%%%%%%%%%%%%%%%%%%%%%%%%%%%%%%%%%%
% University/School Laboratory Report
% LaTeX Template
% Version 3.1 (25/3/14)
%
% This template has been downloaded from:
% http://www.LaTeXTemplates.com
%
% Original author:
% Linux and Unix Users Group at Virginia Tech Wiki 
% (https://vtluug.org/wiki/Example_LaTeX_chem_lab_report)
%
% License:
% CC BY-NC-SA 3.0 (http://creativecommons.org/licenses/by-nc-sa/3.0/)
%
%%%%%%%%%%%%%%%%%%%%%%%%%%%%%%%%%%%%%%%%%

%----------------------------------------------------------------------------------------
%	PACKAGES AND DOCUMENT CONFIGURATIONS
%----------------------------------------------------------------------------------------

\documentclass{article}
\usepackage{float}
\usepackage{pgfplots}
%%%%%%%%%%%%%%%%%%%%%%%%%%%%%%%%%%%%%%%%
\usepackage{tikz}
\usetikzlibrary{chains,shapes.multipart}
\usetikzlibrary{shapes,calc}
\usetikzlibrary{automata,positioning}


%%%%%%%%%%%%%%%%%%%%%%%%%%%%%%%%%%%%%%%%
\usepackage[utf8]{inputenc}
\usepackage[italian]{babel}
\usepackage{graphicx} % Required for the inclusion of images
\usepackage{natbib} % Required to change bibliography style to APA
\usepackage{amsmath} % Required for some math elements 

\setlength\parindent{0pt} % Removes all indentation from paragraphs

\renewcommand{\labelenumi}{\alph{enumi}.} % Make numbering in the enumerate environment by letter rather than number (e.g. section 6)

%\usepackage{times} % Uncomment to use the Times New Roman font

%----------------------------------------------------------------------------------------
%	DOCUMENT INFORMATION
%----------------------------------------------------------------------------------------

\title{{\Large Progetto di Programmazione Dichiarativa} \\ Problema delle matrici non sovrapponibili} % Title

\author{Federico \textsc{Corò}} % Author name

\date{\today} % Date for the report

\begin{document}
\maketitle

\section*{Testo progetto}
\textbf{Input:} quattro numeri interi $N$, $M$, $Q$, $K$\\
\textbf{Problema:} dire se esistono $K$ matrici di dimensione $N\times M$ ($N$ righe $M$ colonne)
che abbiano come elementi numeri interi nell'insieme $\{1,2,3,4,\dots,Q\}$.\\
Le $K$ matrici devono essere a due a due non sovrapponibili. Due matrici sono sovrapponibili se possono essere anche parzialmente sovrapposte.

\paragraph*{Esempio}
Per esempio, con $N=3, M=5, Q=5$:

$$
\begin{array}{ccccc}
4 & 1 & 2 & 3 & 4 \\
2 & 2 & 3 & 1 & 5 \\
1 & 1 & 1 & 2 & 2
\end{array}
$$

è sovrapponibile con
$$
\begin{array}{ccccc}
3 & 1 & 5 & 5 & 1 \\
1 & 2 & 2 & 2 & 1 \\
1 & 2 & 3 & 4 & 4
\end{array}
$$

infatti possono essere scritte in questo modo:
$$
\begin{array}{ccccccc}
4 & 1 & 2 & 3 & 4 &   &   \\
2 & 2 & 3 & 1 & 5 & 5 & 1 \\
1 & 1 & 1 & 2 & 2 & 2 & 2 \\
  &   & 1 & 2 & 3 & 4 & 4
  \end{array}
$$

La sovrapposizione può avvenire su qualsiasi porzione delle matrici,
anche su un ''lato'' o solo su un ''angolo'' della matrice.\\
Per esempio la prima matrice e' sovrapponibile con la matrice

$$
\begin{array}{ccccc}
2 & 4 & 2 & 3 & 2 \\
1 & 2 & 2 & 2 & 1 \\
2 & 1 & 3 & 4 & 1
\end{array}
$$

infatti possono essere scritte in questo modo:
            
$$
\begin{array}{cccccccccc}
4 & 1 & 2 & 3 & 4 &   &   &   &   \\
2 & 2 & 3 & 1 & 5 &   &   &   &   \\
1 & 1 & 1 & 2 & 2 & 4 & 2 & 3 & 2 \\
  &   &   &   & 1 & 2 & 2 & 2 & 1 \\
  &   &   &   & 2 & 1 & 3 & 4 & 1
\end{array}
$$

\section*{Implementazione}

\end{document}